% !TeX spellcheck = da_DK
% Setup document class.
%  This will always be the beamer class, but depending on the use of notes,
%  it can be annotated with the option [notes] or  [notes=only], depending
%  on whether notes should be included, or should be the only thing in
%  the document.

\documentclass[xcolor=table]{beamer}

%%%% USE WHEN PRINING HANDOUTS %%%%%

%\documentclass[xcolor=table, handout]{beamer}
%\usepackage{pgfpages}
%\pgfpagesuselayout{2 on 1}[a4paper,border shrink=5mm]

%%%%%%%


% Setup theme.
%\input{preamble/simpletheme.tex}
\input{preamble/regulartheme.tex}


% Import preamble
%%% Initial things %%%
% Increase number of dimen registers
\usepackage{etex}
% Fix various issues with LaTeX2e
\usepackage{fixltx2e}
% Font package
\usepackage{fourier}


%%% Translations and character encodings %%%
% Enable use of several characters, including æ, ø and å
\usepackage[utf8]{inputenc}
% Danish language
\usepackage[UKenglish]{babel}
% Use PostScript fonts instead of bitmap ones. Also does other stuff.
\usepackage[T1]{fontenc}
% Various LaTeX symbols
\usepackage{latexsym}
% Wider selection of colours
\usepackage{xcolor}
% Improved element justification
\usepackage{ragged2e}
% Font improvements
\usepackage{fix-cm}
% Enables various forms of lines, like double-underlining (\uuline{})
\usepackage{ulem}
% Sets the tolerance for distance between words, determining when to hyphenate.
\pretolerance=2500

\usepackage{rotating}

%%% Figures and tables (Floats) %%%
% Enable multi-rows and -columns
\usepackage{multirow}
\usepackage{multicol}
% Double, horizontal lines
\usepackage{hhline}
% Enables coloured tables
\usepackage{colortbl}
% Prettier tables
\usepackage{booktabs}


%%% Mathematic formulas %%%
% AMS math
\usepackage{amsmath}
\usepackage{amssymb}
% Extra fonts (for math, I think)
\usepackage{stmaryrd}
% Access text symbols
\usepackage{textcomp}
% Extend AMS
\usepackage{mathtools}
\usepackage{cancel}


%%% Graphics %%%
% Various image-commands
\usepackage{eso-pic}
% Use JPEG and PNG images
\usepackage{graphicx}

%%% Code listing %%%
\usepackage{color}
\definecolor{bluekeywords}{rgb}{0.13,0.13,1}
\definecolor{greencomments}{rgb}{0,0.5,0}
\definecolor{redstrings}{rgb}{0.9,0,0}

\usepackage{courier}

\usepackage{listings}

\lstset{language=[Sharp]C,
  captionpos=b,
  columns=fixed,
  numbers=left,
  numberstyle=\tiny,
  showspaces=false,
  showtabs=false,
  tabsize=3,
  breaklines=true,
  inputencoding=utf8,
  showstringspaces=false,
  breakatwhitespace=true,
  escapeinside={(*@}{@*)},
  commentstyle=\color{greencomments},
  keywordstyle=\color{bluekeywords},
  stringstyle=\color{redstrings},
  basicstyle=\ttfamily\small,
}

\lstdefinestyle{make}{tabsize=2}


%%% References, bibtex and URLs %%%
% Post URLs. Allows breaking at hyphens to help avoid long links.
\usepackage{url}
% Better cross references
\usepackage[english]{varioref}
% Define a new 'leo' style for URL package, that will use a smaller font
\makeatletter
\def\url@leostyle{%
  \@ifundefined{selectfont}{\def\UrlFont{\sf}}{\def\UrlFont{\small\ttfamily}}
}
\makeatother
% And of course, use this new style
\urlstyle{leo}

\hypersetup{pdfstartview={Fit}}
%%%%%%%%%%%%%%%%%%%%%%%%%%%%%%%%%%%%%%%%%%%%%%%%
%Flowchart
\usepackage{tikz}
\usepackage{tkz-graph}
\usetikzlibrary{shapes,arrows}
\usepackage{tikz-qtree}
\usepackage{tikzscale}
\usetikzlibrary{shapes.multipart, arrows, matrix, automata, positioning, shadows, decorations.pathreplacing, calc}
%%%%%%%%%%%%%%%%%%%%%%%%%%%%%%%%%%%%%%%%%%%%%%%%



\usepackage{color}
\definecolor{bluekeywords}{rgb}{0.13,0.13,1}
\definecolor{greencomments}{rgb}{0,0.5,0}
\definecolor{redstrings}{rgb}{0.9,0,0}
\usepackage{courier}
\usepackage{listings}

\lstset{
    literate={ø}{{\o}}1
         {æ}{{\ae}}1
         {å}{{\aa}}1
         {Ø}{{\O}}1
         {Æ}{{\AE}}1
         {Å}{{\AA}}1
         {§}{{\S}}1
}

%%% Colour definitions %%%
% Defines: gray
\definecolor{gray}{gray}{0.80}
% Defines: numbercolor
\definecolor{numbercolor}{gray}{0.7}
% Defines: shadecolor
\definecolor{shadecolor}{RGB}{33,26,82}
% Defines: aaublue

\definecolor{aaublue}{RGB}{33,26,82}


\colorlet{punct}{red!60!black}
\definecolor{background}{HTML}{EEEEEE}
\definecolor{delim}{RGB}{20,105,176}
\colorlet{numb}{magenta!60!black}

\newcommand*\rot{\rotatebox{90}}


\usepackage{pifont}
\newcommand{\cmark}{\ding{51}}%
\newcommand{\xmark}{\ding{55}}%

\usepackage{todonotes}


\showboxdepth=5
\showboxbreadth=5


\tikzstyle{decision} = [diamond, draw, fill=yellow!20, text width=4.5em, text badly centered, node distance=3cm, inner sep=3pt]
\tikzstyle{block} = [rectangle, draw, fill=green!40, text width=5em, text centered, rounded corners, minimum height=4em]
\tikzstyle{line} = [draw, -latex']
%\tikzstyle{cloud} = [draw, ellipse,fill=red!20, node distance=3cm, minimum height=2em]
\tikzstyle{cloud_nospace} = [cloud, node distance=1cm]
\tikzstyle{preDefProc} = [draw,rectangle split, rectangle split horizontal,rectangle split parts=3, fill=green!40,minimum height=4em]


\input{preamble/colors.tex}

\usepackage{csquotes}


% Additional settings for boxes
\setbeamercolor{headerCol}{fg=black,bg=lightgray}
\setbeamercolor{bodyCol}{fg=white,bg=gray}
\setbeamercovered{transparent=20}

% Define document stuff
\title[GIRAF]{Software Development in the GIRAF Multi–Project}
\subtitle[Exam]{Exam Presentation}
\author[SW618F16]{SW618F16}
\date{June 20th 2016}

\institute[
%  {\includegraphics[scale=0.2]{aau_segl}}\\ %insert a company, department or university logo
Software (SW6)\\
Aalborg University\\
Denmark
] % optional - is placed in the bottom of the sidebar on every slide
{% is placed on the title page
  Aalborg University\\
  Denmark

  %there must be an empty line above this line - otherwise some unwanted space is added between the university and the country (I do not know why;( )
}

% Specify a logo on the titlepage (you can specify additional logos an include them in
% institute command below
\pgfdeclareimage[height=1.5cm]{titlepagelogo}{AAUgraphics/aau_logo_new} % placed on the title page
%\pgfdeclareimage[height=1.5cm]{titlepagelogo2}{graphics/aau_logo_new} % placed on the title page
\titlegraphic{% is placed on the bottom of the title page
  \pgfuseimage{titlepagelogo}
  %  \hspace{1cm}\pgfuseimage{titlepagelogo2}
}


\begin{document}

{\aauwavesbg
  \begin{frame}[plain,noframenumbering]
    \titlepage
  \end{frame}}

% ==================== SUBJECTS ======================
\section{Introduction}
    \begin{frame}[t]{Introduction}\framesubtitle{The Multi--Project}
        \begin{itemize}
	        \item What is GIRAF?
	        	\begin{itemize}
	        		\item \textbf{G}raphical \textbf{I}nterface \textbf{R}esources for \textbf{A}utistic \textbf{F}olk.
	        	\end{itemize}
        	\item Bachelor project for 6 years now.
        	\item Cooperation of Software Development.
        	\item Minimal Viable Product:
        		\begin{itemize}
        			\item Week Schedule and its prerequisites.
        			\item Voice Game as stand-alone
        			\item Launcher 
        		\end{itemize}
    			\item No synchronisation
    			\item No security
    		\end{itemize}
    \end{frame}

\section{Development Processes}
	\subsection{Description}
	\subsubsection{Multi--Project}
    \begin{frame}[t]{Development Process}\framesubtitle{Cooperating in multiple teams}
    \begin{itemize}
        \item Scrum of Scrum
        	\begin{itemize}
        		\item Weekly Meeting
    			\begin{itemize}
        			\item Knowledge sharing 
        			\item Help seeking
        		\end{itemize}
        		\item Product Owner        			
    			\begin{itemize}
        			\item Important role
        			\item Not everyone in the group participated
        			\item Weak communication with customers
        			\item User stories
        		\end{itemize}
        		\item Scrum Master
        		\begin{itemize}
        			\item Wanted to be PO themselves
        			\item Helped when the PO needed help
        		\end{itemize}
    		\end{itemize}
    \end{itemize}
	\end{frame}
	 \subsubsection{Internal--Project}
	\begin{frame}[t]{Development Process}\framesubtitle{Internal Process}
    \begin{itemize}
        \item Scrum
        \item Daily Meeting
        \item Scrum Board
        \item Pair Programming
        \item Did not always work as planned
    \end{itemize}
	\end{frame}
	\subsubsection{Rest--API Project}
		\begin{frame}[t]{Development Process}\framesubtitle{REST API}
	
    \begin{itemize}
        \item Cooperation sub--project
        \item High Code Quality
        \item Should be easy to overtake
        \item Code Review
    \end{itemize}
	\end{frame}

	\subsection{Evaluation}
		\begin{frame}[t]{Evaluation}\framesubtitle{Evaluating the processes}
	
    \begin{itemize}
        \item Sprint Retrospectives
        \item Development process underwent changes we wanted
        \item Our own thoughts
        	\begin{itemize}
        		\item No \textit{leader} of each app
        		\item Everyone tried to do everything
        		\item Lack of focus in the groups, resulted in weak coherence of a project.
        	\end{itemize}
    \end{itemize}
	\end{frame}
\section{Project Management \& Development}
\subsection{Obstacles}
\begin{frame}{Project Management \& Development}
    \begin{itemize}
        \item Version Control
        \begin{itemize}
            \item Manageing repositories and sourcecode
        \end{itemize}
        \item Code Review
        \item Wiki
        \item Documentation
        \item Continuous Integration
        \item Artifact Repository
    \end{itemize}
\end{frame}
\subsection{Tools}
\begin{frame}[t]{Project Management \& Development}\framesubtitle{Version Control}
    \begin{itemize}
        \item Git
        \begin{itemize}
            \item Foundation of the workflow
            \item Gogs
            \begin{itemize}
                \item Restricted to AAU developers
                \item Custom git--hooks
            \end{itemize}
            \item Powerful
            \item Easy to make mistakes
        \end{itemize}
    \end{itemize}
\end{frame}

\begin{frame}[t]{Project Management \& Development}\framesubtitle{Code Review, Wiki}
    \begin{itemize}
        \item Phabricator
        \begin{itemize}
            \item Arcanist
            \begin{itemize}
                \item Uses git
                \item Interfaces with Phabricator
                \item Unit tests \& linting
            \end{itemize}
            \item Web interface
            \begin{itemize}
                \item Hub for the entire multi--project
                \item User story and task management
                \begin{itemize}
                    \item Backlog
                    \item Assign to groups
                \end{itemize}
                \item Scrumboards
                \item Wiki
                \item Code review
            \end{itemize}
        \end{itemize}
    \end{itemize}
\end{frame}
\begin{frame}[t]{Project Management \& Development}\framesubtitle{CI, Documentation, Artifact Repo}
    \begin{itemize}
        \item Jenkins
        \begin{itemize}
            \item Gradle
            \begin{itemize}
                \item DevOps
            \end{itemize}
            \item Automated Testing
            \begin{itemize}
                \item Monkey test
                \item Unit tests
            \end{itemize}
            \item Artifactory
            \begin{itemize}
                \item Maven repository
                \item Multiple versions of libraries
            \end{itemize}
            \item Javadoc
        \end{itemize}
    \end{itemize}
\end{frame}
\section{Architecture}
\begin{frame}[t]{Architecture}\framesubtitle{Old GIRAF architecture}
    \only<1>{%
    \begin{figure}
        \centering
        \includegraphics[width=0.8\textwidth]{images/old_architecture.png}
    \end{figure}}
    \only<2>{%
    \begin{figure}
        \centering
        \includegraphics[width=0.8\textwidth]{images/old_architecture_memes.png}
    \end{figure}}
\end{frame}

\begin{frame}[t]{Architecture}\framesubtitle{New GIRAF architecture}
    \centering
    \scalebox{0.6}{%
    \input{images/architecture_diagram.tex}}
\end{frame}

\begin{frame}[t]{Architecture}\framesubtitle{New GIRAF architecture}
    \begin{itemize}
        \item Layered architecture
        \item Easier to test
        \item Scaleable
        \item Easier to make other GIRAF clients
    \end{itemize}
\end{frame}

\section{Troels' Slides}
    \begin{frame}[t]{Troels' Slides}\framesubtitle{Troels' Slides}
        Test Text
    \end{frame}
\section{Epilogue}
    \begin{frame}[t]{Epilogue}\framesubtitle{Content Overview}
        \begin{itemize}
            \item Multiproject process
            \begin{itemize}
                \item Work partitioning
                \item Areas of responsibility
                \item REST specific workflow
            \end{itemize}
            \item Future Work
            \begin{itemize}
                \item Remaining design for REST
                \item Utilising REST API
            \end{itemize}
        \end{itemize}
    \end{frame}

    \subsection{Multiproject Process}
    \begin{frame}[t]{Multiproject Process}\framesubtitle{Work Partioning}  
        \begin{itemize}
            \item Initial learning period
            \item No \textit{co--ordinator} for any apps
            \item Several abandoned tasks per sprint
            \item Random high priority selection
            %Bring in example
            \item Priority assignments
            \begin{itemize}
                \item ``what they feel is the best for the project and the customers''
            \end{itemize}
            \item Greater focus on cross group communication
            %Information Sharing
        \end{itemize}
    \end{frame}

    \begin{frame}[t]{Multiproject Process}\framesubtitle{Improving on the App Development Process}
        \begin{itemize}
            \item Areas of responsibility
            \begin{itemize}
                \item Multiple different perceptions
                \item More valuable areas of responsibility
                \begin{itemize}
                    \item A \textit{co-ordinator} for each app in MVP
                \end{itemize}
                \item Remove restrictions on accessability
            \end{itemize}
            \bigskip
            \item Applying REST development values to app development.
            \begin{itemize}
                \item Code-review
                \item Linter
                \item Quality over quantity
                %Technical Debt
            \end{itemize}
    \end{frame}

    \subsection{Future Work}
    \begin{frame}[t]{Future Work}\framesubtitle{Future Work for the REST API}
        \begin{itemize}            
            \item More endpoints
            \item Standalone considerations
            \item Web administration
            %PHP exists, we havnt touched but this should be priotised for next gen
            \item Client-Side implementation
            \begin{itemize}
                \item Visual feedback for live updates
                \item Refactor model
                \item Translate data
            \end{itemize}
        \end{itemize}
    \end{frame}


% ====================================================

% Final slide
{\aauwavesbg
  \begin{frame}[plain,noframenumbering]
    \finalpage{\texttt{throw new PresentationIsOverException();}}
  \end{frame}}

\end{document}
