\section{Implementation}
    \subsection{Automated Code Generation}
    \begin{frame}[t]{Code Generation From Model}\framesubtitle{Model Driven Development}
        Ideally
        \begin{enumerate}
            \item Make model
            \item Verify it
            \item ??? (Automated Code Generation)
            \item Implementation is over
        \end{enumerate}
        \bigskip
        \begin{itemize}
            \item <2->Assuming the 3rd step is correct then it will work, because the model works!
            \item <3->However this is a \textcolor{orange}{difficult} problem, and code generation will be done manually 
        \end{itemize}
    \end{frame}
    \subsection{Implementing for the Arduino}
    \begin{frame}[t]{Implementing for the Arduino}\framesubtitle{Platform and toolchain}
        \begin{itemize}
            \item C/C++-like language
            \item $< 500$ LOC
            \item Uses a Library (RadioHead) to communicate
            \item Relatively simple to make given the model
            \item Hardest problem being synchronization related
            \item Uses 8560 bytes out of 32256 bytes of storage (26 \%)
            \item Uses 672 bytes of 2048 bytes of memory (32 \%)
        \end{itemize}

    \end{frame}
    \begin{frame}[t]{Implementing for the Arduino}\framesubtitle{Usercode}
        \begin{itemize}
            \item \texttt{void userCodeRunonce()}
            \begin{itemize}
                \item Once pr. time-slot
                \item WCET must be less than the time set to execute user code in
            \end{itemize}
            \item \texttt{void userCodeRepeat()}
            \begin{itemize}
                \item Used to fill the rest of the user code execution time
                \item WCET must be less than the guard time before transmitting
            \end{itemize}
            \item \texttt{void userSensorPoll()}
            \begin{itemize}
                \item Can be enabled to poll sensor while receiving
                \item WCET should be short i.e. sub-millisecond
            \end{itemize}
        \end{itemize}

    \end{frame}

\section{Demonstration}
    \subsection{Remote}
    \begin{frame}[t]{Demonstration}\framesubtitle{Remote controlled presentation.}
        \begin{itemize}
            \item 2 devices
            \item Reads from sensor
            \item Sends special byte
            \item Device connected to PC reads
            \item Bash script reads from device
            \item VBScript sends keystroke to PDF-reader
            \item Effect: Can control slides from remote!
            \item (WCET without packetloss: 600 ms * 3 = 1800 ms)
        \end{itemize}
    \end{frame}
    \subsection{Lightshow}
    \begin{frame}[t]{Demonstration}\framesubtitle{Lightshow!}
        \begin{itemize}
            \item 3 devices
            \item Usercode!
            \item Can start multiple at the same time
            \item Green LED on while sending
            \item Red LED on while receiving
            \item Multiple buttons turning on LEDs/actuators
            \item (Note: Timeslot is 2 x desired length for demonstrative purposes.)
        \end{itemize}
    \end{frame}
